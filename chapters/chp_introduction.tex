% !TEX root = manuscript.tex
\glsresetall[\acronymtype]
%\addbibresource{../resource/ref.bib}
\chapter{Introduction} \label{chp:introduction}
\Gls{5g} wireless communication systems are being designed to provide high data and transmission rates, massive device connectivity and enhanced reliability at low latency. \cite{Amin_2016}.
%
To this end, a reliable \gls{la} process for \gls{5g} \gls{nr} is needed for coping with the increased demands in terms of the physical layer performance \cite{chu01}.
%
\Gls{la} is a key technology to keep the \gls{bler} below a predefined threshold while maximizing the throughput.
%
% The \gls{amc} is a key solution used in 4G systems and envisaged to \gls{5g} \gls{nr} system.
%
% This approach consists in selecting the appropriate \gls{mcs} based on the channel quality.
%
% Basically, the systems use an \gls{amc} scheme that selects the appropriate \gls{mcs}.
% One key technique for \gls{la} is the \gls{amc} that selects the appropriate \gls{mcs}.
A very well known approach is the use of \gls{amc} solutions, which aims at controlling the \gls{bler} by adaptively switching among modulation schemes and coding rates based on a \gls{cqi}.
%
They use the channel state information to keep the \gls{bler} below a predefined threshold.
% That is made based on the channel quality in order to keep the \gls{bler} below a predefined threshold.
%
In \gls{lte}, the target \gls{bler} is fixed to 10\%, but the \gls{5g} \gls{nr} will cover a wider spectrum of services, and they impose new set of \gls{bler} targets \cite{Amin_2016,fantacci2009adaptive}.
%
Another aspect in \gls{la} is the so-called \textit{rank adaptation}, which defines the appropriate number of transmitted spatial streams to be selected before transmission.
%
Rank adaptation is used in order to increase the throughput in low interference scenarios and link reliability in high interference scenarios.

%
% %
% %
%
% In this context, the link adaptation technique of \gls{amc} is of great interest.
% %
% \Gls{amc} is a resource allocation technique used in link adaptation that allows the system to select the most appropriate \gls{mcs} to better cope with the changing channel conditions \cite{fantacci2009adaptive}.
% %

\Gls{amc} is a solution to match the modulation scheme and coding rate to the time-varying nature of the wireless channel.
%
Periodically, the \gls{ue} measures the channel quality and processes this information to map into a \gls{cqi}.
%
Typically, each \gls{cqi} represents a \gls{snr} interval \cite{Blanquez-Casado2016}.
%
The \gls{bs} uses the \gls{cqi} reported by the \gls{ue} to define the appropriate \gls{mcs}.
%
Thanks to the \gls{pdcch}, the new \gls{mcs} is informed to the \gls{ue} through the \gls{dci} \cite{ErikDahlman5G}.
%
By its turn, rank adaptation improves the systems performance, especially when used with \gls{irc} by selecting the number of spatially multiplexed data streams.
%
In high interference scenarios, lower ranks are preferred as it improves the interference suppression at the receiver side, while at low interference scenarios higher ranks can be used to increase the throughput \cite{catania2015distributed}.
%
% In the downlink \gls{amc} procedure, the \gls{ue} suggests to the \gls{bs} an appropriate \gls{mcs} in the \gls{amc} set to be used \cite{Sang2014}.
% %
% This proposed \gls{mcs} is informed to the \gls{bs} by means of a \gls{cqi}, typically each \gls{cqi} represents a \gls{snr} interval \cite{Blanquez-Casado2016}.
% %
% In possession of this information, the \gls{bs} selects an \gls{mcs} to transmit and reports  its selection to the \gls{ue}.

The goal of the \gls{la} is an automatic choice of the best parameters depending on the user and applications requirements.
%
As such, \gls{ml} algorithms are well suited to this application, because of their capabilities of learning patterns, forecasting behaviors and generating models \cite{survey-son}.
%
A \gls{ml} category of particular interest to cellular systems is the \gls{rl}, because of its applicability in optimization problems \cite{survey-son}, such as backhaul optimization~\cite{jaber2015}, coverage and capacity optimization~\cite{Fan2014} and resource optimization~\cite{Miozzo2017SwitchOnOffPF}.
%
As such, the \gls{rl} framework has become an attractive tool to devise novel \gls{5g} \gls{la} due to its capacity of solving problems whose model varies over time.
%
% The goal of the \gls{la} is an automatic choice of the best parameters depending on the user, channel conditions and applications requirements.
% While in \gls{lte}, the look-up table provides fixed \gls{amc} rules for all the users, the novel system needs a more flexible one that can be adjusted according to user channel state.
%
% \Gls{rl} falls into a category of \gls{ml} problems, and it has been applied in problems \cite{survey-son} such as backhaul optimization~\cite{jaber2015}, coverage and capacity optimization~\cite{Fan2014} and resource optimization~\cite{Miozzo2017SwitchOnOffPF}.
%
% The use of \gls{rl} in the context of \gls{la} has been recently addressed in \cite{continuousState}, \cite{bruno2014robust} and \cite{DRL_AMC}.

% The goal of the \gls{la} is an automatic choice of the best parameters, in this case the \gls{mcs}, depending on the user and applications requirements.
% %
% As such, \gls{ml} algorithms are well suited to this application, because of their capabilities of learning patterns, forecasting behaviors and generating models \cite{survey-son}.
%
% A \gls{ml} category of particular interest to cellular systems is the \gls{rl}, because of its applicability in optimization problems \cite{survey-son}, such as backhaul optimization~\cite{jaber2015}, coverage and capacity optimization~\cite{Fan2014} and resource optimization~\cite{Miozzo2017SwitchOnOffPF}.


% There are few works that use \gls{rl} in \gls{la} problem.
%
% In \cite{continuousState}, the selection of the \gls{mcs} is based on the received \gls{sinr}, as such the state space is continuous, and the learning algorithm must handle this large state space.
% %
% In \cite{bruno2014robust} a Q-learning approach is used to solve the \gls{amc} problem in the context of a \gls{lte} network.
%
% A deep reinforcement learning approach is used in \cite{DRL_AMC} as a solution to the \gls{amc} problem, in a cognitive heterogeneous network.
%


%%%%%%%%%%%%%%%%%%%%%%%%%%%%%%%%%%%%%%%%%%%%%%%%%%%%%%%%%%%%
\section{State-of-the-Art}
% The use of \gls{rl} in the context of \gls{la} has been recently addressed in \cite{continuousState}, \cite{bruno2014robust} and \cite{DRL_AMC}.

% The goal of the \gls{la} is an automatic choice of the best parameters, in this case the \gls{mcs}, depending on the user and applications requirements.
% %
% As such, \gls{ml} algorithms are well suited to this application, because of their capabilities of learning patterns, forecasting behaviors and generating models \cite{survey-son}.
%
% A \gls{ml} category of particular interest to cellular systems is the \gls{rl}, because of its applicability in optimization problems \cite{survey-son}, such as backhaul optimization~\cite{jaber2015}, coverage and capacity optimization~\cite{Fan2014} and resource optimization~\cite{Miozzo2017SwitchOnOffPF}.

One of the main techniques applied to \gls{la} is the \gls{olla} \cite{Sampath1997}.
%
The goal of the \gls{olla} is to improve the \gls{amc}, since the static \gls{sinr} thresholds of the \gls{illa} do not perform well due to the variations of the channel.
%
Therefore, the \gls{olla} is an additional technique, applied on top of the \gls{illa}, which adjusts the threshold according to the reliability of the transmitted packets \cite{Blanquez-Casado2016}.

But \gls{olla} can experience a slow convergence and degrade the average throughput \cite{10.1145/3341216.3342212}.
%
Several works tried to improve the performance of \gls{olla}.
%
In \cite{Blanquez-Casado2016} the authors propose a variation of \gls{olla} with dynamic step size, instead of the fixed offset given to the \gls{sinr} thresholds adopted in the traditional \gls{olla}.
%
In \cite{Sarret2015}, a dynamic step size approach is also proposed, by operating with a positive fixed offset and an adjustable negative offset.
%
In \cite{Sarret2015}, the proposed solution is also evaluated when used in conjunction with a rank adaptation scheme proposed in \cite{catania2015distributed}.

Regarding \gls{rl} solutions, there are few works that use \gls{rl} in \gls{la} problem.
%
In \cite{continuousState}, the selection of the \gls{mcs} is based on the received \gls{sinr}, as such the state space is continuous, and the learning algorithm must handle this large state space.
%
In \cite{bruno2014robust} a Q-learning approach is used to solve the \gls{amc} problem in the context of a \gls{lte} network.
%
A deep reinforcement learning approach is used in \cite{DRL_AMC} as a solution to the \gls{amc} problem, in a cognitive heterogeneous network.


%%%%%%%%%%%%%%%%%%%%%%%%%%%%%%%%%%%%%%%%%%%%%%%%%%%%%%%%%%%%
\section{Objectives, Contributions and Thesis Structure}

The main objectives of this work are:
\begin{enumerate}
    \item Develop and study the effectiveness of an \gls{rl} solution to \gls{amc} bases on a Q-Learning framework.
    %
    \item Extend the idea to a more complete \gls{la} with rank adaptation.
    %
    \item Present the \gls{phy} concepts necessary to understand the transmission procedures while also giving an understanding of the implementation work made to prepare the simulator for this work.
\end{enumerate}

Chapter \ref{chp:theory} presents an overview of the main \gls{5g} \gls{nr} features and \gls{rl} techniques used in this work.
%
More specifically, it gives a short description of the transmission procedures involved in downlink transmissions and it presents the associated \gls{phy} procedures, while also providing an overview of some fundamental concepts of \gls{rl} and of the Q-learning algorithm.
%
Chapter \ref{chp:amc} introduces the \gls{amc} problem, our proposed solution and compares it against standard \gls{amc} solutions, namely the \gls{illa} and \gls{olla}.
%
Chapter \ref{chp:la} presents a more general \gls{la} problem, which includes both the \gls{mcs} selection and the transmission rank selection, with a detailed explanation of the proposed solution.
%
Simulation results are provided to evaluated the performance of the proposed Q-learning based \gls{amc} and \gls{la} algorithms, considering the \gls{5g} \gls{nr} physical layer.
%
Chapter \ref{chp:conclusion} summarizes the main conclusions of this work.

Therewith, the main contributions of this work are:
\begin{enumerate}
    \item Proposition and analysis of an \gls{amc} and a \gls{la} solution that selects the \gls{mcs} and the \gls{pmi} by using a \gls{rl} framework.
    \item Solutions in compliance with \gls{5g} \gls{nr} specification as we consider the physical layer structure specified in the standard \cite{3gpp.38.212}
    \item The more general \gls{la} solution of Chapter \ref{chp:la} also complies with the \gls{5g} \gls{nr} procedures for data as it considers the multi-antenna precoder matrices from the standard \cite{3gpp.38.214}.
\end{enumerate}
%
% Furthermore, our solution complies with \gls{5g} \gls{nr} physical layer specification as we consider the whole chain of channel coding specified in the standard \cite{3gpp.38.212} while also using the multi-antenna precoder matrices from the standard \cite{3gpp.38.214}.
% As such, the main difference between our work and the previous approaches is the implementation of the channel coding as proposed by the \gls{5g} standard while also using the multi-antenna precoder matrices from the standard \cite{3gpp.38.214}

%%%%%%%%%%%%%%%%%%%%%%%%%%%%%%%%%%%%%%%%%%%%%%%%%%%%%%%%%%%%
\section{Scientific Contributions}
\begin{refsection}

The content of this thesis, more specifically Chapter \ref{chp:amc}, has been partially published and presented at 2019 IEEE Globecom Workshop.
%
A preprint can be found with the following bibliographic information:
%
\begin{itemize}
	\item \printpublication{mota2019adaptive}
\end{itemize}

It is worth mentioning that this master thesis was developed under the context of Ericsson/UFC technical cooperation project entitled UFC.47 - \textit{5G-MAGIC (Machine leArninG lInk Control)} in which a number of 2 technical reports have been delivered.
% \begin{itemize}
% 	\item UFC.47 - \textit{5G-MAGIC (Machine leArninG lInk Control)}, November/2018 - October/2020;
% \end{itemize}
%
% in which a number of 2 technical reports have been delivered. %

\end{refsection}
