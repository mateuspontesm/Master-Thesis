\begin{resumo}[Resumo]
\begin{otherlanguage*}{brazil}

\setlength\parindent{24pt}
Nós projetamos dois \textit{frameworks} auto-exploratórios, baseados no algoritmo Q-learning, para adaptação de enlace.
%
Na primeira parte, a solution projetada permite a \gls{bs} escolher uma \gls{mcs} adequada que maximiza a eficiência espectral enquanto mantém uma baixa \gls{bler}.
%
Neste \textit{framework}, a \gls{bs} escolhe a \gls{mcs} baseado em um reporte do \gls{cqi} pelo \gls{ue}.
%
Então, uma transmissão é feita com a \gls{mcs} escolhida e os resultados dessa transmissão são convertidos em recompensas pela \gls{bs}, que utiliza essas recompensas para aprender um mapeamento adequado entra \gls{cqi} e \gls{mcs}.
%
Comparando com uma tabela \textit{look-up table} e a solução de \textit{\gls{olla}}, a solução proposta alcança performance superior em termos de eficiência espectral e \gls{bler}.

Na segunda parte, nós estudamos diferentes arquiteturas que permitem a \gls{bs} escolher adequadamente a \gls{mcs} e o ranque de transmissão (ou fator de multiplexação espacial) com o objetivo de maximizar o throughput enquanto mantém uma baixa \gls{bler}.
%
Neste \textit{framework}, a \gls{bs} decide a \gls{mcs} e o \gls{pmi} baseado no \gls{cqi} reportado pelo usuário.
%
Uma transmissão é feita com a \gls{mcs} e o \gls{pmi} selecionados e os resultados dessa transmissão são convertidos em recompensa pela \gls{bs}, que são utilizadas para melhorar o mapeamento entre \gls{cqi} para tanto \gls{mcs} quanto \gls{pmi}.
%
O efeito da informação extra provida pelo \gls{ri} também é analisada neste trabalho.
%
Essa solução produz uma performance similar quando comparada com uma solução linha de base \textit{genie-aided}, mesmo com a sinalização reduzida em comparação à sinalização padrão para adaptação de enlace prevista para o 5G NR.

% \indent\indent Sistemas 5G serão baseados na implantação de largos conjuntos de antenas operando no espectro de ondas milimétricas. %
% Com mais antenas e maior largura de banda, a estimação da qualidade do canal e o envio dessas medidas do usuário para a estação rádio base serão processos computacionalmente mais complexos que os atuais. %
% Além disso, devido a uma maior degradação do sinal no espectro de ondas milimétricas, a confiabilidade desses sistemas pode ser um desafio. %
% Neste contexto, a presente tese analisa duas estratégias para tratar ambos os problemas: complexidade e confiabilidade. %
%
% A primeira estratégia consiste em explorar a redução das flutuações do canal devido ao uso de feixes estreitos com largos conjuntos de antenas (o canal “endurece”). %
% Quando este fenômeno ocorre, funções de camadas superiores baseadas em medições podem ser otimizadas. %
%
% A segunda estratégia é relacionada à integração entre sistemas 5G e LTE. %
% Mais precisamente, os usuários têm a capacidade de se conectarem simultaneamente a sistemas de ambas as tecnologias. %
% Isto é chamado conexão dual. %
%
% Antes de abordar essas duas estratégias, apresentamos uma visão geral das principais características do 5G usadas nessa tese e padronizadas pelas especificações do 3GPP versão 15. %
% Depois disso, apresentamos análises gerais relacionadas à conexão dual e ao endurecimento do canal. %
% Finalmente, investigamos esses dois conceitos da perspectiva da alocação de recursos de rádio. %
%
% Mais especificamente, propomos soluções baseadas no endurecimento do canal e relacionadas à medição da qualidade do canal e ao envio destes dados. %
% Além disso, também apresentamos soluções para seleção de estação rádio base e alocação de recursos em sistemas com múltiplas tecnologias e múltiplas conexões. %
% Análises numéricas considerando parâmetros 5G são apresentadas para validar os métodos propostos. %

\vspace*{2ex}
\textbf{Palavras-chave: } aprendizagem por reforço, aprendizagem de máquina, adaptação de enlace, adaptação de posto.

\end{otherlanguage*}
\end{resumo}
