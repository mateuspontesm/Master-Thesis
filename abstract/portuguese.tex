\begin{resumo}[Resumo]
\begin{otherlanguage*}{brazil}

\setlength\parindent{24pt}
Neste trabalho são propostos dois frameworks auto-exploratórios, baseados em aprendizado por reforço para a adaptação de enlace em sistemas de comunicações sem fio 5G.
%
Primeiramente, é apresentada uma solução baseada em Q-learning para modulação e codificação adaptativa que permite a estação base aprender o mapeamento entre o esquema de modulação e codificação e o indicador de qualidade do canal (do inglês CQI- channel quality indicator), visando maximizar a eficiência espectral do sistema.
%
Comparada às soluções clássicas de modulação e codificação adaptativa, a solução proposta alcança desempenho superior em termos de eficiência espectral e taxa de erro de blocos.
%
Na segunda parte deste trabalho, considera-se um problema mais amplo no contexto de sistemas com múltiplas entradas e múltiplas saídas (do inglês, MIMO - multiple-input multiple-output), em que a estação base e o usuário são equipados com arranjos de antenas.
%
Para este sistema, é apresentada uma solução baseada em Q-learning para a seleção conjunta do esquema de modulação e codificação e do número de camadas espaciais de transmissão (fator de multiplexação espacial), bem como o esquema de precodificação.
%
Neste caso, o mapeamento é aprendido baseado nas informações de CQI e do indicador de posto da tramissão (do inglês, RI - rank indicator).
%
De acordo com resultados de simulação, a solução proposta atinge um desempenho similar ao da solução de referência (genie-aided) porém com menor quantidade de sinalização necessária quando comparada à sinalização especificada no padrão do 5G.

%
% Nós projetamos dois \textit{frameworks} auto-exploratórios, baseados no algoritmo Q-learning, para adaptação de enlace.
% %
% Na primeira parte, a solution projetada permite a \gls{bs} escolher uma \gls{mcs} adequada que maximiza a eficiência espectral enquanto mantém uma baixa \gls{bler}.
% %
% Neste \textit{framework}, a \gls{bs} escolhe a \gls{mcs} baseado em um reporte do \gls{cqi} pelo \gls{ue}.
% %
% Então, uma transmissão é feita com a \gls{mcs} escolhida e os resultados dessa transmissão são convertidos em recompensas pela \gls{bs}, que utiliza essas recompensas para aprender um mapeamento adequado entra \gls{cqi} e \gls{mcs}.
% %
% Comparando com uma tabela \textit{look-up table} e a solução de \textit{\gls{olla}}, a solução proposta alcança performance superior em termos de eficiência espectral e \gls{bler}.
%
% Na segunda parte, nós estudamos diferentes arquiteturas que permitem a \gls{bs} escolher adequadamente a \gls{mcs} e o ranque de transmissão (ou fator de multiplexação espacial) com o objetivo de maximizar o throughput enquanto mantém uma baixa \gls{bler}.
% %
% Neste \textit{framework}, a \gls{bs} decide a \gls{mcs} e o \gls{pmi} baseado no \gls{cqi} reportado pelo usuário.
% %
% Uma transmissão é feita com a \gls{mcs} e o \gls{pmi} selecionados e os resultados dessa transmissão são convertidos em recompensa pela \gls{bs}, que são utilizadas para melhorar o mapeamento entre \gls{cqi} para tanto \gls{mcs} quanto \gls{pmi}.
% %
% O efeito da informação extra provida pelo \gls{ri} também é analisada neste trabalho.
% %
% Essa solução produz uma performance similar quando comparada com uma solução linha de base \textit{genie-aided}, mesmo com a sinalização reduzida em comparação à sinalização padrão para adaptação de enlace prevista para o 5G NR.


\vspace*{2ex}
\textbf{Palavras-chave: } aprendizagem por reforço, aprendizagem de máquina, adaptação de enlace, adaptação de posto.

\end{otherlanguage*}
\end{resumo}
