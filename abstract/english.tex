\begin{resumo}

\setlength\parindent{24pt}

In this work we propose two self-exploratory frameworks, based on \gls{rl} for link adaptation in \gls{5g} wireless communication systems.
%
Firstly, a Q-learning solution for \gls{amc} is presented that allows the base station to learn the mapping between the \gls{mcs} and the \gls{cqi}, in order to maximize the spectral efficiency of the system.
%
Compared to classic \gls{amc} solutions, the proposed solution achieves superior performances in terms of spectral efficiency and \gls{bler}.
%
In the second part of this work, a broader problem is considered in the context of \gls{mimo} systems with spatial multiplexing.
%
For this system, a solution based on Q-learning is presented for the joint selection of the \gls{mcs} and the number of spatial transmission layers (spatial multiplexing factor), as well as the precoding scheme.
%
In this case, the mapping is learned based on the information from the \gls{cqi} and the \gls{ri}.
%
According to our simulation results, the proposed solution achieves a performance similar to that of the reference (genie-aided) solution, but with less signaling compared to the one specified in the 5G standard.
%
% We design two self-exploratory \gls{rl} frameworks, based on the Q-learning algorithm, for link adaptation.
% %
% In the first part, the designed solution enables the \gls{bs} to choose a suitable \gls{mcs} that maximizes the spectral efficiency while maintaining a low \gls{bler}.
% %
% In this framework, the BS chooses the \gls{mcs} based on the \gls{cqi} reported by the \gls{ue}.
% %
% A transmission is made with the chosen \gls{mcs} and the results of this transmission are converted by the \gls{bs} into rewards that the \gls{bs} uses to learn the suitable mapping from \gls{cqi} to \gls{mcs}.
% %
% Comparing with a conventional fixed look-up table and the outer loop link adaptation, the proposed framework achieves superior performance in terms of spectral efficiency and \gls{bler}.
%
% In the second part, we study different architectures that enables the BS to choose a suitable MCS and transmission rank (or spatial multiplexing factor) with the goal of maximizing the throughput while maintaining a low BLER.
% %
% In this framework, the BS decides the MCS and the PMI based on the CQI reported by the UE.
% %
% A transmission is made with the selected MCS and PMI and the results of this transmission are converted by the BS into rewards that the BS uses to learn the suitable mapping from CQI to both MCS and PMI.
% %
% The effect of the extra information provided by the RI is also analyzed.
% %
% This solution enables a similar performance when compared to a genie-aided baseline, even with the reduced signaling compared to the standard architecture in NR.

\vspace*{2ex}
\textbf{Keywords: } reinforcement learning, machine learning, link adaptation, rank adaptation.
\end{resumo}
